% ---------- Titelblad Masterproef Faculteit Wetenschappen -----------
% Dit document is opgesteld voor compilatie met pdflatex.  Indien je
% wilt compileren met latex naar dvi/ps, dien je de figuren naar
% (e)ps-formaat om te zetten.
%                           -- december 2012
% -------------------------------------------------------------------
\RequirePackage{fix-cm}

% --------------------- In te laden pakketten -----------------------
% Deze kan je eventueel toevoegen aan de pakketten die je al inlaadt
% als je dit titelblad integreert met de rest van thesis.
% -------------------------------------------------------------------
\usepackage{graphicx,xcolor,textpos}

%----------------------- Custom stuff -------------------------------

\graphicspath{./}
\usepackage{makeidx}
\usepackage{amsmath}
\usepackage{amssymb}
\usepackage[english]{babel}
\usepackage{listings}
\usepackage{eurosym}
\usepackage{import}
\usepackage{multirow}
\usepackage{tabularx}
\usepackage{fancybox}
\usepackage{csquotes}
\usepackage{pdfpages}
\usepackage{xspace}
\usepackage{hyperref}
\usepackage[acronym]{glossaries}
\usepackage[automake]{glossaries-extra}
\usepackage{float}
\usepackage{booktabs}


\usepackage{wrapfig}
\usepackage[backend=biber,style=numeric]{biblatex}
\addbibresource{literatur.bib}


%------------------------ Plot packages ----------------------------
\usepackage{pgfplots}
\usepackage{tikz}
\usepackage{nicematrix}

\usetikzlibrary{positioning,arrows,calc,matrix, fit}

\usetikzlibrary{external}
%\tikzset{external/system call={lualatex -shell-escape -halt-on-error -interaction=batchmode -jobname "\image" "\texsource"}}
%\tikzexternalize[prefix=figures/]
\pgfplotsset{compat=1.18}
%
%%\usepackage[utf8]{inputenc}
%\usepackage{pgf}
%\usepackage{units}
%\usepackage{metalogo}
%\usepackage{caption}
%\usepackage{subcaption}
%\usepackage[mode=buildnew]{standalone}% requires -shell-escape

\DeclareUnicodeCharacter{2212}{−}
\usepgfplotslibrary{groupplots,dateplot}
\usetikzlibrary{patterns,shapes.arrows}
\pgfplotsset{compat=newest}

%----------------------- Comments ----------------------------------
\definecolor{tabBlue}{HTML}{1f77b4}
\definecolor{tabOrange}{HTML}{ff7f0e}
\definecolor{tabGreen}{HTML}{2ca02c}
\definecolor{tabRed}{HTML}{d62728}
\definecolor{tabPurple}{HTML}{9467bd}
\definecolor{tabBrown}{HTML}{8c564b}
\definecolor{tabPink}{HTML}{e377c2}
\definecolor{tabGray}{HTML}{7f7f7f}
\definecolor{tabOlive}{HTML}{bcbd22}
\definecolor{tabCyan}{HTML}{17becf}

\newcommand{\CS}[1]{{\color{tabBlue} {\bf (CS: #1)}}}
\newcommand{\FS}[1]{{\color{tabOrange} {\bf (FS: #1)}}}
\newcommand{\MW}[1]{{\color{tabGreen} {\bf (MW: #1)}}}

\makeindex